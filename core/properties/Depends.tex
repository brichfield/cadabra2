\cdbproperty{Depends}{\it comma separated list of indices,
coordinates, derivatives, accents}

Makes an object implicitly dependent on other objects, i.e.~assumes
that the indicated object is a function of the arguments of the
property. For example,
\begin{screen}{1,2}
x::Coordinate.
\phi::Depends(x).
\end{screen}
makes $\phi$ an implicit function of $x$. Instead of indicating the
coordinate on which the object depends, it is also possible to
indicate which derivatives would yield a non-zero answer, as in
\begin{screen}{1,2}
\nabla{#}::Derivative.
\phi::Depends(\nabla).
\end{screen}
Finally, it is possible to use an index name to indicate on which
coordinates a field depends,
\begin{screen}{1,2}
{m,n,p,q}::Indices(vector).
\phi::Depends(m).
\end{screen}

Taking objects out of derivatives (because they do not depend on them)
is handled using the \subscommand{unwrap} algorithm.

If you want to make an object depend on more than one thing, you need
to specify them all in one \subsprop{Depends} property. If you specify
them in two separate properties, the last property will overwrite
the previous one. Therefore, you get
\begin{screen}{1,2,3,4,5,6}
\hat{#}::Accent.
\partial{#}::PartialDerivative.
A::Depends(\hat).
A::Depends(\partial).
\hat{A};
@unwrap!(%);
A;
\end{screen}
instead of \verb|\hat{A}| which you might have expected.

\cdbseealgo{unwrap}
\cdbseeprop{Derivative}

