\cdbproperty{SigmaMatrix}{}

These are the invariant tensors relating the~$(\tfrac{1}{2},\tfrac{1}{2})$ to the vector representation of
SO(3,1). \Cdb uses the Wess~\& Bagger conventions,
which means that the metric has signature~$\eta = {\rm
  diag}(-1,1,1,1)$ and
\begin{equation}
(\sigma^{\mu})_{\alpha\dot{\beta}} = ( -{\mathbb 1}, \vec\sigma )_{\alpha\dot{\beta}}\,,\quad
(\bar{\sigma}^{\mu})^{\dot{\alpha}\beta} = (-{\mathbb 1}, -\vec\sigma)^{\dot{\alpha}\beta}\,.
\end{equation}
When the objects carry two vector indices, they are understood to be
\begin{equation}
(\sigma^{m n})_{\alpha}{}^{\beta} \equiv \frac{1}{4}( \sigma^m
  \bar{\sigma}^n - \sigma^n \bar{\sigma}^m)_{\alpha}{}^{\beta}\,,\quad\quad
(\bar{\sigma}^{m n})^{\dot{\alpha}}{}_{\dot{\beta}} \equiv
  \frac{1}{4}(\bar{\sigma}^m \sigma^{n}
- \bar{\sigma}^n \sigma^{m})^{\dot{\alpha}}{}_{\dot{\beta}}\,.
\end{equation}
See below for algorithms dealing with the conversion from indexed to
index-free notation.
